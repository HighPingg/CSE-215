\documentclass[12pt]{article}
\usepackage{enumitem}
\usepackage{amssymb}
\author{Vincent Zheng}

\begin{document}

\title{CSE 215 - Homework 4}
\maketitle

\section*{Problem Set 1}
\begin{enumerate}[label = (\alph*)]
    \item

        \textbf{Basis Step}: Prove P(1) is true. \\
        \\
        $1^3$ = 1 \\
        $[\frac{1(1+1)}{2}]^2$ = $\frac{2}{2}^2$ = $1^2$ = 1 \\
        \\
        1 = 1, therefore this statement holds for P(1)\\
        \\
        \textbf{Induction Step}: Assume P(k) is true for some k$\geq$1. Prove P(k+1)
                                 is true. \\
        \\
        $(1^3 + 2^3 + ... + k^3) + (k + 1)^3$ \\
        = $[\frac{k(k+1)}{2}]^2 + (k + 1)^3$ \hspace{3em} ($\because$ P(k) is true) \\
        = $\frac{k(k+1)}{2} \frac{k(k+1)}{2} + k^3 + 3k^2 + 3k + 1$ ($\because$ Expansion) \\
        = $\frac{k^2 (k^2 + 2k + 1)}{4} + \frac{4k^3 + 12k^2 + 12k + 4}{4}$ \\
        = $\frac{k^4 + 2k^3 + k^2}{4} + \frac{4k^3 + 12k^2 + 12k + 4}{4}$ \\
        = $\frac{k^4 + 6k^3 + 13k^2 + 12k + 4}{4}$ \\
        $\ast$ Find the roots of $k^4 + 6k^3 + 13k^2 + 12k + 4$ with graphing $\ast$ \\
        = $\frac{(k + 1)^2 (k + 2)^2}{4}$ \hspace{3em} ($\because$ Factoring the numerator)\\
        = $\frac{(k + 1) (k + 2)}{2}\frac{(k + 1) (k + 2)}{2}$ \hspace{3em} ($\because$
            Factoring Out Perfect Squares)\\
        = $[\frac{(k + 1)(k + 2)}{4}]^2$  \hspace{3em} ($\because$ Combining the 2 Fractions) \\
        \\
        $\therefore$ P(k + 1) is true. \\

    \item
    
        \textbf{Basis Step}: Prove P(0) is true. \\
        \\
        $\sum_{i=1}^{0 + 1} i\times2^i = 1\times2^1 = 2 = 2$ \\
        $0\times 2^{0+2}+2 = 0+2 = 2$ \\
        \\
        2 = 2, therefore this statement holds true fore P(0). \\
        \\
        \textbf{Induction Step}: Assume P(k) is true for some k$\geq$0. Prove P(k+1)
                                 is true. \\
        \\
        $\sum_{i=1}^{k + 1} i\times2^i$ \\
        = $(\sum_{i=1}^{k + 1} i\times2^i) + (k + 2)\times 2^{k+2}$ (Taking Out the k + 2 term) \\
        = $k\times 2^{k+2}+2 + (k + 2)\times 2^{k+2}$ \hspace{3em} ($\because$ P(k) is true) \\
        = $(2k + 2)2^{k+2}+2$ \hspace{3em} ($\because$ Combining Like Terms) \\
        = $(k + 1)(2)(2^{k+2})+2$ \hspace{3em} ($\because$ Factoring Out 2) \\
        = $(k + 1)(2^{k+1+2})+2$ \hspace{3em} ($\because$ Exponent Rule) \\
        \\
        $\therefore$ P(k + 1) is true.

    \item

        \textbf{Basis Step}: Prove P(0) is true. \\
        \\
        $\prod_{i=0}^{0} (\frac{1}{2i+1})(\frac{1}{2i+2}) = (\frac{1}{1})(\frac{1}{2}) =
            \frac{1}{2}$ \\
        $\frac{1}{(2(0)+2)!} = \frac{1}{2}$\\
        \\
        $\frac{1}{2} = \frac{1}{2}$, therefore this statement holds true for P(0). \\
        \\
        \textbf{Induction Step}: Assume P(k) is true for some k$\geq$0. Prove P(k+1)
                                 is true. \\
        \\
        $\prod_{i=0}^{k+1} (\frac{1}{2i+1})(\frac{1}{2i+2})$ \\
        = $(\prod_{i=0}^{k} (\frac{1}{2i+1})(\frac{1}{2i+2}))(\frac{1}{2(k+1)+1})
            (\frac{1}{2(k+1)+2})$ (Taking out the k + 1 term) \\
        = $\frac{1}{(2k+2)!}\cdot\frac{1}{2k+3}\cdot\frac{1}{2k+4}$ \hspace{3em} ($\because$ P(k)
            is true) \\
        = $\frac{1}{(2k+4)!}$ \hspace{3em} ($\because$ Merging the Denominators) \\
        = $\frac{1}{(2k+2+2)!} = \frac{1}{(2(k+1)+2)!}$
        \\
        $\therefore$ P(k + 1) is true.

    \newpage
    \item

        \textbf{Basis Step}: Prove P(0) is true. \\
        \\
        $0^3-7(0)+3 = 3$ \\
        \\
        3 is divisible by 3, therefore this statement holds true for P(0). \\
        \\
        \textbf{Induction Step}: Assume P(k) is true for some k$\geq$0. Prove P(k+1)
                                 is true. \\
        \\
        $(k+1)^3-7(k+1)+3$ \\
        = $(k+1)(k^2+2k+1)-7k-7+3$ \hspace{3em} ($\because$ Expanding $(k+1)^2$) \\
        = $k^3+3k^2+3k+1-7k-7+3$  \hspace{3em} ($\because$ Expaniding $(k+1)(k^2+2k+1)$) \\
        = $(k^3-7k+3)+(3k^2+3k+1-7)$ \hspace{3em} ($\because$ Reordering the Terms) \\
        = $(k^3-7k+3)+3(k^2+k-2)$ \hspace{3em} ($\because$ Factoring Out a 3) \\
        \\
        $(k^3-7k+3)$ is divisible by 3 because we assumed this in the induction step. Therefore
        it can be written as (3x) where x is an integer. \\
        \\
        = $3x + 3(k^2+k-2)$ \hspace{3em} ($\because$ P(k) is true) \\
        = $3(x+k^2+k-2)$ \hspace{3em} ($\because$ Factored Out a 3) \\
        = 3 $\cdot$ integer \hspace{3em}
        \\
        $\therefore$ P(k + 1) is true.
    
    \item 
    
        \textbf{Basis Step}: Prove P(0) is true. \\
        \\
        1 + 3(0) = 1 \\
        $4^0$ = 1 \\
        1 is $\leq$ 1, therefore the statement holds for P(0). \\
        \\
        \textbf{Induction Step}: Assume P(k) is true for some k$\geq$0. Prove P(k+1)
                                 is true. \\
        \\
        $1+3(k+1)$ \\
        = $1+3k+3$ \hspace{3em} ($\because$ Distributive Property) \\
        $\leq$ $4^k+3$ \hspace{3em} ($\because$ P(k) is true) \\
        $\leq$ $4^k+3\cdot 4^{k}$ \hspace{3em} ($\because$ 3 $\leq3\cdot 4^{k}$ for all integer
            k $\geq$ 0) \\
        = $4\cdot 4^k = 4^{k+1}$ \hspace{3em} ($\because$ Adding the 2 Terms and Exponent Rules) \\
        $\therefore$ P(k + 1) is true.\\
    
    \item

        \textbf{Basis Step}: Prove P(2) is true. \\
        \\
        $\sqrt{2} \approx 1.414$ \\
        $\frac{1}{\sqrt{1}}\frac{1}{\sqrt{2}} \approx 1.707$ \\
        \\
        1.414 $>$ 1.707, thus this statement holds for P(2). \\
        \\
        \textbf{Induction Step}: Assume P(k) is true for some k $\geq$ 2. Prove P(k+1)
                                is true. \\
        \\
        $\frac{1}{\sqrt{1}}+\frac{1}{\sqrt{2}}$ + ... + $\frac{1}{\sqrt{k}}+\frac{1}{\sqrt{k+1}}$ \\
        $>\sqrt{k} + \frac{1}{\sqrt{k+1}}$ \hspace{5em} ($\because$ P(k) is true) \\
        = $\frac{\sqrt{k}\sqrt{k+1}}{\sqrt{k+1}} + \frac{1}{\sqrt{k+1}}$ \hspace{3em} ($\because$
            Multiply top and bottom by $\frac{\sqrt{k+1}}{\sqrt{k+1}}$) \\
        = $\frac{\sqrt{k^2+k}}{\sqrt{k+1}} + \frac{1}{\sqrt{k+1}}$ \hspace{3em} ($\because$
            Merge the two radicals) \\
        = $\frac{\sqrt{k^2+k} + 1}{\sqrt{k+1}}$ \hspace{5em} ($\because$ Adding Fractions) \\
        $\geq \frac{\sqrt{k^2} + 1}{\sqrt{k+1}}$ \hspace{6em} ($\because \sqrt{k^2+k} \geq
            \sqrt{k^2}, k \geq 2$) \\
        = $\frac{k + 1}{\sqrt{k+1}}$ \hspace{7em} ($\because$ Simplifying $\sqrt{k^2}$) \\
        = $\frac{(k+1)\sqrt{k+1}}{k+1}$ \hspace{6em} ($\because$ Multiply top and bottom by
            $\frac{\sqrt{k+1}}{\sqrt{k+1}}$) \\
        = $\sqrt{k+1}$ \hspace{7em} ($\because$ Divide top and bottom by $k + 1$) \\
        \\
        $\therefore$ P(k + 1) is true.


    \item 

        \textbf{Basis Step}: Prove P(2) is true. \\
        \\
        $1 + 2x$ \\
        $(1 + x)^2 = x^2 + 2x + 1$ \\
        \\
        $2x + 1 \leq x^2 + 2x + 1$ for all $x > -1$, thus the statement holds for P(2).\\
        \\
        \textbf{Induction Step}: Assume P(k) is true for some k $\geq$ 2. Prove P(k+1)
                                is true. \\
        $(1 + x)^{k+1}$ \\
        = $(1 + x)(1 + x)^k$ \hspace{5em} ($\because$ Exponent Rules) \\
        $\geq (1 + x)(1 + kx)$ \hspace{5em} ($\because$ P(k) is true) \\
        = $1 + x + kx + kx^2$ \hspace{5em} ($\because$ Expanding the Multiplication) \\
        = $1 + (1 + k)x + kx^2$ \hspace{5em} ($\because$ Factoring Out x) \\
        $\geq 1 + (1 + k)x$ \hspace{7em} ($\because kx^2 \geq 0$) \\
        \\
        $\therefore$ P(k + 1) is true.
    
    \item

        \textbf{Basis Step}: Prove P(2) is true. \\
        \\
        $\frac{2(2-1)}{2} = \frac{2}{2} = 1$
        \\
        Since with 2 people, you can only have 1 handshake, this statement holds for P(2). \\
        \\
        \textbf{Induction Step}: Assume P(k) is true for some k $\geq$ 2. Prove P(k+1)
                                is true. \\
        \\
        If someone else arrives, he would have to shake k hands, therefore we can just add k to
        P(k), our induction hypothesis of $\frac{k(k-1)}{2}$. \\
        \\
        $\frac{k(k-1)}{2} + k$ \\
        = $\frac{k(k-1) + 2k}{2}$ \hspace{6em} ($\because$ Adding k to the fraction) \\
        = $\frac{k^2 - k + 2k}{2}$ \hspace{6em} ($\because$ Distributing the k) \\
        = $\frac{k^2 + k}{2}$ \hspace{6em} ($\because$ Simplifying) \\
        = $\frac{k(k + 1)}{2}$ \hspace{6em} ($\because$ Factoring Out the k) \\
        \\
        $\therefore$ P(k + 1) is true.

    \item 
    
        \textbf{Basis Step}: Prove P(3) is true. \\
        \\
        $\frac{3(3-3)}{2} = 0$
        \\
        Since a triangle doesn't have any diagonals, this statement holds for P(3). \\
        \\
        \textbf{Induction Step}: Assume P(k) is true for some k $\geq$ 3. Prove P(k+1)
                                is true. \\
        \\
        If another side is added, it would have add k - 1 more diagonals, therefore we can just
        add k - 1 to P(k), our induction hypothesis of $\frac{k(k-3)}{2}$. \\
        \\
        $\frac{k(k-3)}{2} + k - 1$ \\
        = $\frac{k(k-3) + 2(k-1)}{2}$ \hspace{6em} ($\because$ Adding k - 1 to the fraction) \\
        = $\frac{k^2 - 3k + 2k - 2}{2}$ \hspace{6em} ($\because$ Distributing the k and 2) \\
        = $\frac{k^2 - k - 2}{2}$ \hspace{6em} ($\because$ Simplifying) \\
        = $\frac{(k - 2)(k + 1)}{2}$ \hspace{6em} ($\because$ Factoring Out the k) \\
        \\
        $\therefore$ P(k + 1) is true.

    \item 
    
        \textbf{Basis Step}: Prove P(1) is true. \\
        \\
        1! = 1 \\
        \\
        Since 1 number can only be organized in 1 way, and 1! = 1, the statement holds for P(1).
        \\
        \textbf{Induction Step}: Assume P(k) is true for some k $\geq$ 1. Prove P(k+1)
                                is true. \\
        \\
        If we have k object, through the induction hyposthesis, there are $k!$ possible permutations
        of the objects, now if we add 1 more object, we can find all of the possible permutations
        of $k+1$ by inserting the number into every possible slot (which is k + 1 possible slots). \\
        \\
        $k!(k + 1)$ \\
        = $(k+1)!$ \hspace{6em} ($\because$ By Definition) \\
        \\
        $\therefore$ P(k + 1) is true.

\end{enumerate}

\section*{Problem Set 2}
\begin{enumerate}[label = (\alph*)]
    \item 
        \begin{enumerate}[label = \alph*.]
            \item

                \textbf{Basis Step}: Prove P(0), P(1), P(2) is true. \\
                \\
                $3^0 = 1$ \\
                $3^1 = 3$ \\
                $3^2 = 9$
                \\
                Since $1 \leq 1$, $2 \leq 3$, and $3 \leq 9$, this statement holds for the
                basis cases.
                \\
                \textbf{Induction Step}: Assume P(i) is true for some k $\geq$ 2 and any 
                                         $i \in [0, k]$. Prove P(k+1) is true. \\
                \\
                $h_{k+1} = h_{k} + h_{k-1} + h_{k-2}$ \hspace{3em} ($\because$ By Definition) \\
                $\leq 3^{k} + 3^{k-1} + 3^{k-2}$ \hspace{4em} ($\because$ P(i) is true) \\
                = $9 \cdot 3^{k-2} + 3 \cdot 3^{k-2} + 3^{k-2}$ \hspace{3em} ($\because$ Exponent Rules) \\
                = $(9+3+1) \cdot 3^{k-2}$ \hspace{5em} ($\because$ Factor Out $3^{k-3}$) \\
                $\leq 27 \cdot 3^{k-2}$ \hspace{8em} ($\because$ 27 $>$ 13) \\
                = $3^k+1$ \hspace{10em} ($\because$ Exponent Rules) \\
                \\
                $\therefore$ P(k+1) is true.

            \item 

                \textbf{Basis Step}: Prove P(0), P(1), P(2), P(3) is true. \\
                \\
                Since $s > 1.83$, $h_2 \leq s^2$, $h_3 \leq s^3$, and $h_4 \leq s^4$, this statement holds
                for the basis cases. \\
                \\
                \textbf{Induction Step}: Assume P(i) is true for some k $\geq$ 2 and any 
                                        $i \in [0, k]$. Prove P(k+1) is true. \\
                \\
                $h_{k+1} = h_{k} + h_{k-1} + h_{k-2}$ \hspace{3em} ($\because$ By Definition) \\
                $\leq s^{k} + s^{k-1} + s^{k-2}$ \hspace{4em} ($\because$ P(i) is true) \\
                = $s^2 \cdot s^{k-2} + s \cdot s^{k-2} + s^{k-2}$ \hspace{3em} ($\because$ Exponent Rules) \\
                = $(s^2+s+1) \cdot s^{k-2}$ \hspace{5em} ($\because$ Factor Out $s^{k-3}$) \\
                $\leq s^3 \cdot s^{k-2}$ \hspace{8em} ($\because s^3 > s^2+s+1$) \\
                = $s^{k+1}$ \hspace{10em} ($\because$ Exponent Rules) \\
                \\
                $\therefore$ P(k+1) is true.
        \end{enumerate}
    
    \item

        \textbf{Basis Step}: Prove P(3) and P(4) is true.\\
        \\
        $\frac{7}{4}^3 \approx 5.359$ \\
        $a_3 = a_2 + a_1 = 3 + 1 = 4$ \\
        $\frac{7}{4}^4 \approx 9.378$ \\
        $a_4 = a_3 + a_2 = 4 + 3 = 7$ \\
        \\
        Since $4 \leq 5.359$ and $7 \leq 9.378$, this statement holds for the basis case.\\
        \\
        \textbf{Induction Step}: Assume P(i) is true for some k $\geq$ 3 and any 
        $i \in [0, k]$. Prove P(k+1) is true. \\
        \\
        $a_{k+1} = a_{k} + a_{k-1}$ \\
        $\leq \frac{7}{4}^{k} + \frac{7}{4}^{k-1}$ \hspace{6em} ($\because$ P(i) is true) \\
        = $\frac{7}{4} \cdot \frac{7}{4}^{k-1} + \frac{7}{4}^{k-1}$ \hspace{6em} ($\because$ Exponent Rules) \\
        = $(\frac{7}{4} + 1)\frac{7}{4}^{k-1}$ \hspace{6em} ($\because$ Factoring Out $\frac{7}{4}^{k-2}$) \\
        $\leq (\frac{7}{4})^2 \cdot \frac{7}{4}^{k-1}$ \hspace{6em} ($\because
            (\frac{7}{4})^2 > \frac{7}{4} + 1$) \\
        = $\frac{7}{4}^{k+1}$ \hspace{10em} ($\because$ Exponent Rules) \\
        \\
        $\therefore$ P(k+1) is true.
    
    \item 

        \textbf{Basis Step}: Prove P(1) is true.\\
        \\
        $2^1$ is a circle with 2 people and if we go clockwise from 1, we eliminate 2 and we're left with
        1. Thus, P(1) is true. \\
        \\
        \textbf{Induction Step}: Assume P(i) is true for some k $\geq$ 1 and any 
        $i \in [0, k]$. Prove P(k+1) is true. \\
        \\
        Case 1: If k + 1 is even, then we would eliminate dots until there are 2 dots left, in which case
        the second dot would be eliminated. \\
        Case 2: If k + 1 is odd, then we would eliminate dots until there are 2 dots left, in which case
        the second dot would be eliminated. \\
        \\
        Since both cases lead to 1 dot left, P(k + 1) is true.

    \item 
        
        \textbf{Basis Step}: Prove P(1) and P(2) is true.\\
        \\
        If r = 1, then it can be written as $c_0 = 1$, and $1 \cdot 3^0 = 1$. \\
        If r = 2, then it can be written as $c_0 = 2$, and $1 \cdot 3^0 = 2$. \\
        \\
        \textbf{Induction Step}: Assume P(i) is true for some k $\geq$ 2 and any 
        $i \in [0, k]$. Prove P(k+1) is true.
        \\
        If we assume P(k) to be true, then every number can be written as a multiple of 3 plus either 0,
        1, or 2. Since we can generate 0, 1 or 2, P(k + 1) is true.

    \item 
            
        \textbf{Basis Step}: Prove P(1) is true.\\
        \\
        $F_{3}F_{k+1} - F_{k+2}^2$ \\
        \\
        \textbf{Induction Step}: Assume P(i) is true for some k $\geq$ 2 and any 
        $i \in [0, k]$. Prove P(k+1) is true. \\
        \\
        $F_{k+3}F_{k+1} - F_{k+2}^2$ \\
        = $(F_{k+2}+F_{k+1})F_{k+1} - F_{k+2} \cdot F_{k+2}$ \hspace{3em} ($\because$ Definition of Fibonacci
            Sequence) \\
        = $F_{k+2}F_{k+1} + F_{k+1}^2 - F_{k+2}(F_{k+1}+F_{k})$ \hspace{3em} ($\because$ Definition of Fibonacci
            Sequence) \\
        = $F_{k+2}F_{k+1} + F_{k+1}^2 - F_{k+2}F_{k+1} - F_{k+2}F_{k}$ \hspace{3em} ($\because$ Distributive Property) \\
        = $F_{k+1}^2 - F_{k+2}F_{k}$ \hspace{3em} ($\because$ Subtraction) \\
        = $-(F_{k+2}F_{k} - F_{k+1}^2)$ \hspace{3em} ($\because$ Factor Out -1) \\
        = $-(-1)^k$ \hspace{3em} ($\because$ P(k) is true) \\
        = $(-1)^{k+1}$ \hspace{3em} ($\because$ Exponent Rules) \\
        $\therefore$ P(k + 1) is true.

    \item 

        \textbf{Basis Step}: Prove P(0) is true.\\
        \\
        Since $0^2 = 0$ and f(0) = 0, the statement holds for P(0). \\
        \\
        \textbf{Induction Step}: Assume P(i) is true for some k $\geq$ 0 and any 
        $i \in [0, k]$. Prove P(k+1) is true. \\
        \\
        If k is even, then k + 1 would be odd. \\
        \\
        $f(k+1-1)+2k-1$ \\
        = $f(k)+2k-1$ \hspace{3em} ($\because$ Addition) \\
        = $k^2+2k-1$ \hspace{4em} ($\because$ P(k) is true) \\
        = $(k+1)^2$ \hspace{5em} ($\because$ Factoring) \\
        \\
        $\therefore$ P(k + 1) is true.
\end{enumerate}

\section*{Problem Set 3}
\begin{enumerate}[label = (\alph*)]
    \item 

        \begin{itemize}
            \item [28.]

                $F_{k+1}^2 - F_k^2 - F_{k-1}^2$ \\
                = $(F_k + F_{k-1})^2 - F_k^2 - F_{k-1}^2$ \\
                = $F_k^2 + 2F_kF_{k-1} + F_{k-1}^2 - F_k^2 - F_{k-1}^2$ \\
                = $2F_kF_{k-1}$ \\ QED

            \item [29.]

                $F_{k+1}^2 - F_k^2$ \\
                = $(F_{k} + F_{k-1})^2 - F_k^2$ \\
                = $F_{k}^2 + 2F_{k}F_{k-1} + F_{k-1}^2 - F_k^2$ \\
                = $2F_{k}F_{k-1} + F_{k-1}^2$ \\
                \\ 
                $F_{k-1}F_{k+2}$ \\
                = $F_{k-1}(F_{k+1}+F_{k})$ \\
                = $F_{k-1}F_{k+1}+F_{k-1}F_{k}$ \\
                = $F_{k-1}(F_{k}+F_{k-1})+F_{k-1}F_{k}$ \\
                = $F_{k-1}F_{k}+F_{k-1}^2+F_{k-1}F_{k}$ \\
                = $2F_{k-1}F_{k}+F_{k-1}^2$ \\ QED

        \end{itemize}

    \item 
        \begin{itemize}
            \item [2b.]
                $1 + 3 + 3^2 + ... + 3^{n-2} + 3^{n-1} = \frac{3^{n+1}-1}{3-1} - 3^n = \frac{3^{n+1}-1}{2}-3^n$
            \item [2d.]
                $2^n - 2^{n-1} + 2^{n-2} - 2^{n-3} + ... + (-1)^{n-1} \cdot 2 + (-1)^n$
                = $\frac{2^{n+1}-1}{1} - \frac{2^n-1}{1}$
        \end{itemize}

    \item 
        \begin{itemize}
            \item [9.]
                $\frac{1}{{2}^{n}-1}$
            \item [14.]
                $3\cdot (n) - 1$
            \item [15.]
                $3\cdot (n-1)^2 - n$
        \end{itemize}
\end{enumerate}

\end{document}